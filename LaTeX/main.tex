%% ----------------------------------------------------------------------------
%% LaTeX template based on Elsevier's elsarticle Bundle (LPPL 1.2+).
%% Original template: Elsevier High Energy Astrophysics Journal Template
%%   https://www.overleaf.com/latex/templates/elsevier-high-energy-astrophysics-journal-template/tvmwymmkqvks
%% Copyright 2007-2020 Elsevier Ltd. This project uses the elsarticle class.
%% ----------------------------------------------------------------------------

%\documentclass[preprint,12pt,authoryear]{elsarticle}

%% Use the option review to obtain double line spacing
%% \documentclass[authoryear,preprint,review,12pt]{elsarticle}

%% Use the options 1p,twocolumn; 3p; 3p,twocolumn; 5p; or 5p,twocolumn
%% for a journal layout:
%% \documentclass[final,1p,times,authoryear]{elsarticle}
%% \documentclass[final,1p,times,twocolumn,authoryear]{elsarticle}
%% \documentclass[final,3p,times,authoryear]{elsarticle}
%% \documentclass[final,3p,times,twocolumn,authoryear]{elsarticle}
%% \documentclass[final,5p,times,authoryear]{elsarticle}
 \documentclass[final,5p,times,twocolumn,authoryear]{elsarticle}

%% For including figures, graphicx.sty has been loaded in
%% elsarticle.cls. If you prefer to use the old commands
%% please give \usepackage{epsfig}

%% The amssymb package provides various useful mathematical symbols
\usepackage{amssymb}
\usepackage{amsmath}
\usepackage{lipsum}
\usepackage{tabularx}
\usepackage{booktabs}
\usepackage{hyperref}

%% Use normal list spacing (Pandoc emits \tightlist for compact lists)
\providecommand{\tightlist}{}
%% Compact itemize to match enumerate density
\usepackage{enumitem}
\setlist{nosep}
%% The amsthm package provides extended theorem environments
%% \usepackage{amsthm}

%% The lineno packages adds line numbers. Start line numbering with
%% \begin{linenumbers}, end it with \end{linenumbers}. Or switch it on
%% for the whole article with \linenumbers.
%% \usepackage{lineno}

%% You might want to define your own abbreviated commands for common used terms, e.g.:
\newcommand{\kms}{km\,s$^{-1}$}
\newcommand{\msun}{$M_\odot}

\journal{High Energy Astrophysics}


\begin{document}

\begin{frontmatter}

%% Title, authors and addresses

%% use the tnoteref command within \title for footnotes;
%% use the tnotetext command for theassociated footnote;
%% use the fnref command within \author or \affiliation for footnotes;
%% use the fntext command for theassociated footnote;
%% use the corref command within \author for corresponding author footnotes;
%% use the cortext command for theassociated footnote;
%% use the ead command for the email address,
%% and the form \ead[url] for the home page:
%% \title{Title\tnoteref{label1}}
%% \tnotetext[label1]{}
%% \author{Name\corref{cor1}\fnref{label2}}
%% \ead{email address}
%% \ead[url]{home page}
%% \fntext[label2]{}
%% \cortext[cor1]{}
%% \affiliation{organization={},
%%            addressline={}, 
%%            city={},
%%            postcode={}, 
%%            state={},
%%            country={}}
%% \fntext[label3]{}

\title{Title of paper}

%% use optional labels to link authors explicitly to addresses:
%% \author[label1,label2]{}
%% \affiliation[label1]{organization={},
%%             addressline={},
%%             city={},
%%             postcode={},
%%             state={},
%%             country={}}
%%
%% \affiliation[label2]{organization={},
%%             addressline={},
%%             city={},
%%             postcode={},
%%             state={},
%%             country={}}

\author[first]{Author name}
\affiliation[first]{organization={University of the Moon},%Department and Organization
            addressline={}, 
            city={Earth},
            postcode={}, 
            state={},
            country={}}

\begin{abstract}
%% Text of abstract
Example abstract for the High Energy Astrophysics Journal. Here you provide a brief summary of the research and the results.
\end{abstract}

%%Graphical abstract
%\begin{graphicalabstract}
%\includegraphics{grabs}
%\end{graphicalabstract}

%%Research highlights
%\begin{highlights}
%\item Research highlight 1
%\item Research highlight 2
%\end{highlights}

\begin{keyword}
%% keywords here, in the form: keyword \sep keyword, up to a maximum of 6 keywords
keyword 1 \sep keyword 2 \sep keyword 3 \sep keyword 4

%% PACS codes here, in the form: \PACS code \sep code

%% MSC codes here, in the form: \MSC code \sep code
%% or \MSC[2008] code \sep code (2000 is the default)

\end{keyword}


\end{frontmatter}

%\tableofcontents

%% \linenumbers

%% ----------------------------------------------------------------------------
%% Main text: Content between the markers below is auto-generated by the
%% Obsidian→LaTeX conversion scripts. Do not edit that section manually.
%% ----------------------------------------------------------------------------
% === BEGIN MARKDOWN CONTENT ===
\section{Introduction}\label{example-paper--introduction}

Lorem ipsum dolor sit amet, consectetur adipiscing elit. Sed in rhoncus odio. Nulla sem mauris, ornare eu tempor id, consequat id erat. Sed ut condimentum felis. Praesent sodales eget libero a lacinia. Mauris fermentum augue a felis egestas scelerisque. Etiam venenatis quis tellus vitae vestibulum. Nulla a neque lorem. Sed blandit vitae ante sit amet rutrum. Donec quam diam, finibus a dolor vitae, tempus volutpat odio \hyperref[example-paper--some-header]{Some header}.

\begin{equation}
x = \frac{-b \pm \sqrt{b^2 - 4ac}}{2a}
\end{equation}

Vivamus facilisis commodo ultrices. Quisque bibendum dolor vel tincidunt iaculis. Proin ut iaculis tortor. Suspendisse lacinia, dolor quis tempus dignissim, magna dolor pulvinar tortor, id faucibus nunc tellus a nisl. Proin ultrices, sapien eget blandit consectetur, nibh lacus venenatis dolor, rhoncus varius metus metus elementum mauris. Praesent vel vehicula purus. Nunc euismod dapibus est, vel porttitor urna dictum vitae. Maecenas mattis erat non nisl sagittis, non lacinia justo semper. Aliquam erat volutpat. Mauris placerat ut turpis ut condimentum. Integer gravida tempor diam vel blandit. Ut volutpat porttitor arcu, eu pretium sem elementum sit amet. Morbi accumsan risus at gravida hendrerit. Pellentesque elit dui, suscipit ac molestie laoreet, scelerisque et tortor. Orci varius natoque penatibus et magnis dis parturient montes, nascetur ridiculus mus. Nulla pellentesque commodo tellus sit amet tincidunt.

\section{Some header}\label{example-paper--some-header}

Integer dictum diam est, non malesuada nisl tempor nec. This is an inline equation: \(y = wx + b\) inside a sentence. Nam mollis egestas felis a mattis. Sed tincidunt dolor nibh, at dictum lorem rhoncus eget. Curabitur non est sodales, tincidunt nisl faucibus, viverra dolor. Etiam et euismod arcu, ut tristique massa. Vivamus ut tellus eget nunc finibus venenatis. Interdum et malesuada fames ac ante ipsum primis in faucibus. Mauris sed condimentum lacus, quis ultrices orci. Mauris ac tincidunt ex, vel cursus odio. Praesent maximus augue ut viverra varius in Table~\ref{tbl:animals-table-id}.


\begin{table}[htbp]
  \centering
  \begin{tabularx}{\linewidth}{@{}llll@{}}
    \toprule\noalign{}
Animal & Sound & Legs & Habitat \\
\midrule
Dog & Woof & 4 & House \\
Cat & Meow & 4 & House \\
Cow & Moo & 4 & Farm \\
Duck & Quack & 2 & Pond \\
Snake & Hiss & 0 & Jungle \\
\bottomrule
  \end{tabularx}
  \caption{A label of the animals table}
  \label{tbl:animals-table-id}
\end{table}


Nunc lectus ligula, lacinia vitae pellentesque in, gravida ut turpis. Cras in maximus neque. Donec viverra ut neque ut aliquet. Pellentesque habitant morbi tristique senectus et netus et malesuada fames ac turpis egestas. Integer viverra accumsan dictum. Vestibulum dictum purus vitae lectus vehicula volutpat. Aenean bibendum lacus sit amet odio bibendum laoreet.

\subsection{Another level header}\label{example-paper--another-level-header}

Duis luctus enim luctus enim tempus, sit amet sagittis metus pulvinar. Vestibulum ante ipsum primis in faucibus orci luctus et ultrices posuere cubilia curae; Proin ultrices non augue feugiat scelerisque. Nullam bibendum mauris at lorem efficitur rutrum. In hac habitasse platea dictumst. Morbi sed dolor odio. Vestibulum malesuada justo vitae gravida eleifend. Nam semper et ipsum et bibendum. Duis id turpis orci. Fusce molestie efficitur magna, vel pellentesque magna venenatis et. Nunc vitae purus convallis, porta purus vel, sollicitudin est.

\begin{equation}
\int_{0}^{\infty} e^{-x} \, dx = 1
\end{equation}

Cras pretium felis odio, vel semper sem aliquet viverra. Nunc id ex commodo, pulvinar lectus in, egestas massa. Donec at ipsum at ante vulputate commodo nec nec sapien. Nullam placerat scelerisque nunc a accumsan. Quisque at tortor urna. Fusce a finibus sapien. Vivamus nec eleifend massa. Fusce pretium malesuada nisl, id convallis dui pellentesque at. Donec placerat, turpis vel commodo facilisis, odio felis elementum purus, pharetra tincidunt velit risus in tellus. Etiam vel velit blandit enim lobortis viverra.

\begin{figure}[htbp]
    \centering
    \includegraphics[width=\columnwidth]{figures/not-wrong.jpg}
    \caption{A label of the figure.}
    \label{fig:not-wrong}
\end{figure}


Quisque lectus nulla, ullamcorper id nisl in, gravida ullamcorper quam Figure~\ref{fig:not-wrong}. Vivamus porttitor eros sed efficitur porttitor. Donec quis lacus erat. Curabitur blandit ornare metus, id fermentum massa dictum pharetra. Sed condimentum diam tortor, quis pulvinar est maximus quis. In et ex aliquam, tempus tortor non, venenatis lorem. Nullam sit amet tellus lobortis tortor sodales efficitur. Maecenas vel \citet{sweere2022deep} pretium nunc. Etiam ullamcorper dolor nec erat convallis, ut accumsan sapien tincidunt. Integer vitae mi nec purus eleifend fringilla. Mauris sed viverra risus. Cras hendrerit scelerisque risus, a placerat purus varius eu. Sed lobortis mi ante, vitae aliquam orci rutrum in.

\begin{equation}
A =
\begin{bmatrix}
1 & 2 \\
3 & 4
\end{bmatrix}
\end{equation}

\subsection{And some other header}\label{example-paper--and-some-other-header}

Lorem ipsum dolor sit amet, consectetur adipiscing elit. Phasellus dictum lorem mi, sit amet iaculis ante vehicula a. Aliquam erat volutpat. Ut tincidunt erat vel gravida aliquam. Morbi molestie pellentesque varius. Suspendisse eu lectus ut lacus euismod lobortis. Quisque luctus dui nisl, feugiat lobortis libero laoreet ac. Ut eget neque vel tortor pretium gravida. Nunc elementum maximus diam a tincidunt. In eros purus, laoreet vitae augue et, semper gravida erat. Proin convallis ipsum non tempor ultricies. Ut id lorem risus \hyperref[example-paper--another-level-header]{Another level header}.

\subsubsection{Subsubheader}\label{example-paper--subsubheader}

In hendrerit efficitur efficitur. Donec nec faucibus nisl, at viverra eros. Praesent id egestas enim, ac porttitor enim. Morbi a rutrum ligula, non vulputate nisl. Phasellus sollicitudin id tortor eget gravida. Interdum et malesuada fames ac ante ipsum primis in faucibus. In congue erat libero, eget porttitor felis dapibus at.

\section{Another first level header}\label{example-paper--another-first-level-header}

Fusce commodo, massa et egestas venenatis, orci nibh condimentum massa, id porttitor ipsum metus at lacus. Etiam maximus orci id risus feugiat euismod. Duis maximus justo non lacinia scelerisque. Aenean sapien arcu, auctor ut nisi sed, elementum aliquet ante. Aliquam sem metus, auctor id orci vel, congue tincidunt tellus. Phasellus rutrum neque augue, ac elementum eros luctus sit amet. Suspendisse vestibulum leo velit, a iaculis diam gravida congue. In vel sapien leo. Vivamus cursus nec augue id maximus.

\begin{equation}
\begin{aligned}
y &= wx + b \\
  &= w(ax_1 + bx_2) + b \\
  &= w a x_1 + w b x_2 + b
\end{aligned}
\end{equation}

Morbi quis lacus sapien. Morbi vitae dui nunc. Nam odio libero, dapibus quis elit eget, posuere sodales quam. Pellentesque a tortor in justo ultrices rhoncus sit \citet{vojtekova2021learning} amet iaculis lorem. Sed venenatis, tellus non dictum molestie, sem enim imperdiet elit, tempus congue nisi mauris sed nisi. Aenean dapibus pretium augue eu hendrerit. Suspendisse cursus ullamcorper euismod. Etiam imperdiet viverra ante, vitae congue mi tempor eu.

\begin{equation}
\frac{\partial f}{\partial x} = 2x + 3y
\end{equation}

\subsection{More headers}\label{example-paper--more-headers}

Etiam malesuada tortor erat, eget sollicitudin sem pellentesque sit amet. Vestibulum posuere ex in dolor eleifend, vitae cursus arcu tempor. Vivamus eget tempus orci. Sed tortor nisi, egestas vitae viverra vel, vestibulum sed orci. Proin non magna tellus. Phasellus dignissim tellus et lectus pretium mollis. Praesent nec augue eget lorem ornare dignissim. Pellentesque habitant morbi tristique senectus et netus et malesuada fames ac turpis egestas. Nulla feugiat, nibh at maximus feugiat, urna sapien sollicitudin purus, ut venenatis dui enim eu quam. Praesent in dolor ut lacus tristique ornare. In tempus semper erat id dictum. Aenean maximus fermentum augue ut scelerisque.

\subsubsection{Grocery List}\label{example-paper--grocery-list}

\begin{itemize}
\item
  Apples
\item
  Bread
\item
  Milk
\item
  Cheese
\item
  Chocolate
\end{itemize}
\bigskip
\subsubsection{Weekend Plan}\label{example-paper--weekend-plan}

\begin{enumerate}
\def\labelenumi{\arabic{enumi}.}
\item
  Sleep in
\item
  Go for a walk
\item
  Watch a movie
\item
  Order pizza
\end{enumerate}
\bigskip
Nam luctus vehicula pulvinar \citep{sweere2022deep, vojtekova2021learning}. Maecenas imperdiet ullamcorper velit, ut eleifend tortor luctus vulputate. Curabitur vitae massa vitae quam varius pulvinar. Integer rhoncus rutrum ornare. Proin a aliquet tellus. Suspendisse iaculis vel lacus at interdum. Sed et mollis odio. In accumsan mi eu tortor auctor molestie. Nulla fermentum, mi sit amet vehicula vulputate, diam sem pulvinar nulla, sit amet pulvinar nunc ipsum nec metus. Quisque ut nunc accumsan, vestibulum felis nec, facilisis arcu. Curabitur cursus nulla sed nunc viverra, quis faucibus purus hendrerit. Pellentesque convallis eros vitae venenatis semper. Morbi ultrices malesuada eros. Nulla erat odio, aliquet eget arcu non, vehicula placerat enim. Phasellus pharetra, ex sit amet tincidunt molestie, purus tortor egestas felis, quis fermentum neque lectus sit amet mauris.

Aenean fringilla enim ac egestas fermentum. Sed eget sem mi. Nulla libero massa, dictum et ex vel, posuere facilisis ligula. Nullam suscipit lacus dolor, in laoreet sapien efficitur tincidunt. Suspendisse eget euismod ex. Suspendisse fermentum pellentesque semper. Fusce porta tempus elit, eget dignissim felis pharetra quis. Fusce odio odio, faucibus non fringilla in, ullamcorper non elit. Proin quis iaculis enim, eu dignissim arcu. Proin eros est, varius eu augue eu, malesuada efficitur diam. Donec gravida enim eu libero interdum, vitae ullamcorper risus luctus. Nunc in elit vulputate, bibendum nulla id, porttitor sem. Cras interdum efficitur venenatis. Nulla sodales mollis ligula vitae scelerisque.

Aliquam ullamcorper dui ac vehicula porttitor. Morbi luctus lectus at dui iaculis vestibulum dignissim et eros. Morbi cursus commodo magna a vulputate. Mauris interdum tellus in viverra congue. Etiam leo eros, tristique ut arcu sed, rhoncus finibus eros. Proin rhoncus diam urna, nec ullamcorper mauris sodales sit amet. Quisque lacus nunc, tristique at rutrum at, laoreet id nisi.
% === END MARKDOWN CONTENT ===  


\bibliographystyle{elsarticle-harv}
\bibliography{bibliography}

%% else use the following coding to input the bibitems directly in the
%% TeX file.

%%\begin{thebibliography}{00}

%% \bibitem[Author(year)]{label}
%% For example:

%% \bibitem[Aladro et al.(2015)]{Aladro15} Aladro, R., Martín, S., Riquelme, D., et al. 2015, \aas, 579, A101


%%\end{thebibliography}

\end{document}

\endinput
%%
%% End of file `elsarticle-template-harv.tex'.
